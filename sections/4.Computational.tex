\documentclass[../main.tex]{subfiles}
% DOCUMENT

% \makeatletter
% \def\env@cases{%
%   \let\@ifnextchar\new@ifnextchar
%   \left\lbrace
%   \def\arraystretch{1}%
%   \array{@{}l@{\quad}l@{}}}
% \makeatother

\begin{document}
\chapter{Các thử nghiệm}\label{cuxe1c-thux1eed-nghiux1ec7m}

Để tiến hành đánh giá hiệu năng thuật toán của các chương trước, khóa luận dùng bộ dữ liệu được sinh ngẫu nhiên. 
Chi tiết về cách tạo dữ liệu được trình bày dưới đây.
Thuật toán được cài đặt và
chạy bằng ngôn ngữ C++17, trên máy tính để bản với hệ điều hành Fedora
39, CPU Intel i7-8700 (12) @ 4.600GHz và 128GB RAM. Việc chạy được thực hiện
với các flag tối ưu hóa \texttt{-O2} và thư viện Boost 1.73.0



\section{Tập dữ liệu}\label{tux1eadp-dux1eef-liux1ec7u}

Dữ liệu được chuẩn bị cho việc chạy các thuật toán. Có 4 kiểu dữ liệu
tương ứng với 3 kiểu mạng \(D=(N,A)\) và tập các đỉnh
\(N =\{1,2,\dots, n\}\);\\
\textbf{Kiểu 1:} \(A = \{(i, j) \in N \times N | i < j \}\);

  \begin{itemize}
    \tightlist
    \item
      Mỗi cặp đỉnh bất kỳ đều được nối với nhau bằng một cung.
    \item
      Mật độ của mạng: \(0.5\) (mỗi đỉnh kết nối đến tất cả các đỉnh khác).
    \end{itemize}
\textbf{Kiểu 2:}
  \(A = \{(i, i+1) | i = 1, \dots, n-1\} \cup \{(i, j) \in N \times N | i < j + 1 \text{ và } U_{i, j} < 0.5\}\)
  với mỗi \((i, j)\), giá trị \(U_{i, j}\) được lấy độc lập từ phân phối
  chuẩn \([0, 1]\) hay \(U_{i, j} \sim U[0,1]\)

  \begin{itemize}
    \tightlist
    \item
      Bao gồm nửa số cung có trong \textbf{Kiểu 1},
    \item
      Để đảm bảo luôn có ít nhất một đường đi khả thi, mọi cung giữa các
      đỉnh liên tiếp vẫn được giữ lại.
    \item
      Mật độ mạng: khoảng \(0.25\).
    \end{itemize}
\textbf{Kiểu 3:} \(A = \{(i, j) \in N \times N | i < j < i + 4 \}\);

  \begin{itemize}
    \tightlist
    \item
      Mỗi đỉnh được nối với ba đỉnh tiếp theo theo số thứ tự.
    
      \begin{itemize}
      \tightlist
      \item
        Ví dụ: đỉnh $1$ sẽ nối với đỉnh $2$, $3$ và $4$.
      \end{itemize}
    \item
      Mật độ mạng: khoảng \(3/n\).
    \end{itemize}
% \item
%   \textbf{Kiểu 4:}
%   \(A = \{(i, j+1) | i = 1, \dots, n-1\} \cup \{(i, j) \in N \times N | i < j +1 \text{ và } U_{i, j} < 1/ |j-i|\}\)
%   với \(U_{i, j} \sim U[0,1]\) cho mỗi \((i, j)\).


% \textbf{Kiểu 1:}



% \textbf{Kiểu 2:}



% \textbf{Kiểu 3:}



% \textbf{Kiểu 4:}

% \begin{itemize}
% \tightlist
% \item
%   Khả năng tồn tại một cung giữa hai đỉnh phụ thuộc vào khoảng cách giữa
%   chúng.
% \item
%   Càng xa nhau về mặt số thứ tự thì càng giảm khả năng xuất hiện cung.
% \item
%   Để đảm bảo luôn có ít nhất một đường đi khả thi, tất cả các cung giữa
%   các đỉnh liên tiếp vẫn được giữ lại.
% \item
%   Mật độ mạng: khoảng \(\log n!/(n^2)\).
% \end{itemize}

\section{Hàm vận tốc}\label{huxe0m-thux1eddi-gian}

Để đánh giá mức độ ảnh hưởng của hàm vận tốc đến hiệu năng
của thuật toán, có hai loại hàm được sử dụng. \autoref{fig:11} minh họa cho mỗi
loại. Hàm vận tốc được tạo ra như sau: với mỗi cung
\((i,j)\), sử dụng nội suy tuyến tính từng khúc của một hàm \(f\) (mô tả
chi tiết ở dưới) với các điểm là số nguyên.

\textbf{Loại 1:} \(f\) là đa thức bậc $4$:

\begin{multline*}
   f\big([0, \frac{T}{4}, \frac{T}{2}, \frac{3T}{4}, T]\big) \\
   =
   \begin{cases} 
    [1.6, 1, 1.05, 1, 1.6](B_{i,j} \frac{|j − i|}{10}), & \text {nếu } U_{i,j} < \frac{1}{3}, \\ 
    [2, 1, 1.5, 1, 2](B_{i,j} \frac{|j − i|}{10}), & \text {nếu } \frac{1}{3} \leq U_{i,j} < \frac{2}{3}, \\ 
    [2.5, 1, 1.75, 1, 2.5](B_{i,j} \frac{|j − i|}{10}), & \text{ còn lại},
    \end{cases}
\end{multline*}
    Với \(B_{i, j}, U_{i, j} \sim U[0, 1]\).

\textbf{Loại 2:} \(f\) là đa thức bậc $6$:
\begin{multline*}
    f\big([0, \frac{T}{6}, \frac{T}{3}, \frac{T}{2}, \frac{2T}{3}, \frac{5T}{6}, T]\big) \\
    = 
    \begin{cases} 
    [1, 1.6, 1, 1.05, 1, 1.6, 1](B_{i,j} \frac{|j − i|}{10}), & \text {nếu }  U_{i,j} < \frac{1}{3}, \\ 
    [1, 2, 1, 1.5, 1, 2, 1](B_{i,j} \frac{|j − i|}{10}), & \text {nếu } \frac{1}{3} \leq U_{i,j} < \frac{2}{3}, \\ 
    [1, 2.5, 1, 1.75, 1, 2.5, 1](B_{i,j} \frac{|j − i|}{10}), & \text{ còn lại},
    \end{cases}
\end{multline*}

Với \(B_{i, j}, U_{i, j} \sim U[0, 1]\).

% \textbf{Loại 3:} 
% \(f(t) = (j − i) + sin (B_{i,j} \times t), \text{ với } B_{i,j} \sim U[0, 1].\)

% \begin{figure}
% \centering
% \includegraphics{images/Figure11.png}
% \caption{Figure 11}
% \label{fig:11}
% \end{figure}

\begin{figure}
  \centering

  \subcaptionbox{Loại 1\label{fig:11a}}{\includegraphics[width=.45\textwidth]{edited-images/Figure11a.jpg}}
  \subcaptionbox{Loại 2\label{fig:11b}}{\includegraphics[width=.45\textwidth]{edited-images/Figure11b.jpg}}\\

  % \subcaptionbox{Loại 3\label{fig:11c}}{\includegraphics[width=.45\textwidth]{edited-images/Figure11c.jpg}}

  \caption{Biểu diễn hai loại hàm vận tốc}
  \label{fig:11}
\end{figure}

\section{Nguồn gốc của dữ liệu}\label{nguux1ed3n-gux1ed1c-cux1ee7a-cuxe1c-huxe0m-thux1eddi-gian}

Hàm \textbf{Loại 1} được lấy ý tưởng từ \cite{figliozzi2012time} về
việc chia 5 khoảng thời gian bằng nhau, mỗi khoảng có tốc độ di chuyển
không đổi. Tốc độ di chuyển cơ sở (lấy ngẫu nhiên) được nhân với các hệ
số lần lượt là \(1.6, 1.0, 1.05, 1.0\) và \(1.6\) tại các thời điểm
\(0, \frac 1 4 T, \frac 1 2 T, \frac 3 4 T\) và \(T\) tương ứng. Sau đó sử dụng nội suy đa
thức qua các điểm này để tạo nên hàm liên tục. Cuối cùng chuyển chúng về
dạng hàm tuyến tính từng khúc bằng cách nội suy tuyến tính từng khúc của
đa thức trên tại các điểm nguyên.

Hàm \textbf{Loại 2} được mở rộng từ \textbf{Loại 1} bằng cách thêm hai
điểm ở đầu và cuối để tăng độ khó (tăng bậc).

% Hàm \textbf{Loại 3} gây ra nhiều thách thức cho thuật toán vì chúng
% không đồng pha trên các cung có nhiều điểm cực tiểu và cực đại cục bộ.

Ngoài ra các hàm khác được đề xuất trong \cite{figliozzi2012time} cũng
được thử nghiệm, nhưng thuật toán này tìm ra nghiệm chỉ sau vài lần lặp.
Các nghiệm tối ưu thường bắt đầu ở thời điểm sớm nhất hoặc muộn nhất có
thể, do đó không giúp ích cho việc phân tích hiệu năng của thuật toán.

Hiệu năng của cả bài toán với phương pháp DDD phụ thuộc rất nhiều
vào việc tính toán UTT: có thể tính toán một cách hiệu quả thời gian tối
thiểu di chuyển qua các cung ngay khi bắt đầu chạy thuật toán. Bảng tra
cứu (hash map) là một cấu trúc dữ liệu hiệu quả để lưu trữ các giá trị của việc tính toán trước khi chạy.

\section{Kịch bản và cài
đặt}\label{kux1ecbch-bux1ea3n-vuxe0-cuxe0i-ux111ux1eb7t}

Phần này sẽ bàn về ảnh hưởng của việc chọn BP đến hiệu năng của
thuật toán.

Với bài toán MDP, giả sử cận dưới đang là \(LB^t\) và \(t^+\) là thời
gian đến đỉnh \(n\) của ABSPT ngay sau \(\overline{\mathcal{B}}^t\). Kí
hiệu cho nút trong \(\overline{\mathcal{B}}^t\) là \((i,t_i)\) và trong
\(\overline{\mathcal{B}}^{t^+}\) là \((i, t_i^+)\). Chọn cung \((i,j)\)
sao cho mỗi cung thuộc \(\overline{\mathcal{B}}^{t}\) có BP trong
khoảng thời gian \((t_i, t_i^+)\) và có chỉ số nhỏ nhất (khi xét các
đỉnh từ \(1\) đến \(n\)). Dưới đây là ba cách chọn BP cho
\(c_{i,j}(t)\) với \(t\in (t_i, t_i^+)\):

\begin{enumerate}
\def\labelenumi{\arabic{enumi}.}
\tightlist
\item
  Chọn điểm có giá trị \(c_{i, j}(t)\) nhỏ nhất với
  \(t\in (t_i, t_i^+)\) (\(MIN\))
\item
  Chọn điểm trung vị (trong tập các BP sắp xếp theo thời gian)
  (\(MED\))
\item
  Chọn một điểm bất kì (\(RAND\))
\end{enumerate}

% Với bài toán MTTP, ta so sánh hai kịch bản chọn BP để thêm vào
% danh sách \(L^i\) ở mỗi lần lặp như sau:

% \begin{itemize}
% \tightlist
% \item
%   \textbf{Thêm một BP (S):} Thêm một BP cho một trong các
%   đường con ở mỗi lần lặp.
% \item
%   \textbf{Thêm nhiều BP (M):} Thêm một BP cho mọi đường
%   con ở mỗi lần lặp.
% \end{itemize}

% Quy tắc chọn BP như sau: Để chọn điểm cần thêm vào, ta xác định
% một chuỗi các \(mangrove\) hoàn chỉnh được sử dụng bởi đường đi UTT kí
% hiệu \(P_{LB}\). Mỗi \(mangrove\) hoàn chỉnh được tạo ra từ một BP, giả
% sử đó là điểm \((i,t)\). Khi đó, nếu \((i,t^+)\) là điểm ngay sau
% \((i,t)\) trong \(L^i\) và có một BP chưa được giải quyết tại đỉnh \(i\)
% có thời gian giữa \(t\) và \(t^+\), thì điểm \((i,t^+)\) này sẽ xem xét
% để chọn. Tập hợp các điểm như vậy gọi là tập ứng viên. Cách chọn điểm từ
% tập ứng viên thuộc một trong ba cách ở trên (\(MIN, MED, RAND\)). 
% Với
% kịch bản \textbf{(S)}, thêm một BP của đỉnh \(i\) cho \(i-mangrove\)
% hoàn chỉnh đầu tiên được dùng bởi đường đi UTT từ \(1\) đến \(n\), nếu
% BP ứng viên tồn tại. Trong kịch bản \textbf{(M)}, thêm một BP vào mỗi
% \(mangrove\) hoàn chỉnh được dùng bởi đường đi UTT, nếu BP ứng viên
% tương ứng tồn tại.

\textbf{Đánh giá hiệu năng của việc lựa chọn BP:}

Với bài toán MDP, kết quả được biểu diễn ở \autoref{fig:12} bằng biểu
đồ hộp với các tỉ số: thời gian chạy, số lần lặp và số BP được giải
quyết cho cách chọn \(MIN\), \(MED\) và \(RAND\).

% \begin{figure}
% \centering
% \includegraphics{images/Figure12.png}
% \caption{So sánh các cách chọn với 120 mẫu của bài toán MDP cho \(n=20, T=50\)}
% \label{fig:12}
% \end{figure}

\begin{figure}
  \centering
  \subcaptionbox{Thời gian chạy\label{fig:12a}}{\includegraphics[width=.45\textwidth]{boxplot/Figure_1.png}}
  \subcaptionbox{Số lần lặp\label{fig:12b}}{\includegraphics[width=.45\textwidth]{boxplot/Figure_2.png}}\\

  \subcaptionbox{Số BP được giải quyết\label{fig:12c}}{\includegraphics[width=.45\textwidth]{boxplot/Figure_3.png}}
  \caption{So sánh các cách chọn với 60 mẫu của bài toán MDP cho \(n=30, T=20\)}
  \label{fig:12}
\end{figure}

\begin{itemize}
\tightlist
\item
  \textbf{Biểu đồ hộp:}

  \begin{itemize}
  \tightlist
  \item
    Hình thoi màu xanh biển biểu thị giá trị trung bình.
  \item
    Đường ngang màu đỏ biểu thị giá trị trung vị.
  \end{itemize}
\item
  \textbf{Trục Y:}~thang đo.
\end{itemize}

Qua quan sát các yếu tố trung vị, trung bình và hộp, cách chọn \(MED\) cho kết quả tốt nhất trong cả ba tiêu
chí hiệu năng (thời gian chạy, số lần lặp và số BP được giải quyết). 
Do đó, \(MED\) là lựa chọn ưu tiên trong ba cách.

% Tiếp theo, ta đánh giá tác động của cách chọn BP và số BP được xử lý
% trong mỗi lần lặp khi giải bài toán MTTP.

% \begin{figure}
% \centering
% \includegraphics{images/Figure13.png}
% \caption{So sách các cách chọn với 80 mẫu của bài toán MTTP cho
% \(n=20, T=50\)}
% \label{fig:13}
% \end{figure}

% \begin{figure}
%   % \begin{subfigure}{.5\linewidth}
%   %   % \centering
%   %   \includegraphics{edited-images/Figure13a.jpg}
%   %   \caption{Thời gian chạy}
%   %   \label{fig:13a}
%   % \end{subfigure}
%   % \hfill
%   % \begin{subfigure}{.5\linewidth}
%   %   % \centering
%   %   \includegraphics{edited-images/Figure13b.jpg}
%   %   \caption{Số lần lặp}
%   %   \label{fig:13b}
%   % \end{subfigure}\\[1ex]
%   % \begin{subfigure}{\linewidth}
%   %   \centering
%   %   \includegraphics{edited-images/Figure13c.jpg}
%   %   \caption{Số BP được giải quyết}
%   %   \label{fig:13c}
%   % \end{subfigure}

%   \centering
%   \subcaptionbox{Thời gian chạy\label{fig:13a}}{\includegraphics[width=.45\textwidth]{edited-images/Figure13a.jpg}}
%   \subcaptionbox{Số lần lặp\label{fig:13b}}{\includegraphics[width=.45\textwidth]{edited-images/Figure13b.jpg}}\\

%   \subcaptionbox{Số BP được giải quyết\label{fig:13c}}{\includegraphics[width=.45\textwidth]{edited-images/Figure13c.jpg}}
%   \caption{So sánh các cách chọn với 80 mẫu của bài toán MTTP cho \(n=20, T=50\)}
%   \label{fig:13}
% \end{figure}

% \autoref{fig:13} là biểu đồ hộp của các tỷ số: thời gian chạy, số lần lặp và số
% BP được giải quyết cho các kịch bản: (\(RAND\), \textbf{S}), (\(RAND\),
% \textbf{M}), (\(MED\), \textbf{M}), (\(MIN\), \textbf{S}), (\(MIN\),
% \textbf{M}) khi so sánh với kịch bản (\(MED\), \textbf{S}). Chỉ các mẫu
% với hai hàm vận tốc được sử dụng vì thời gian chạy với hàm
% thứ ba trở đi là rất lớn (sẽ đề cập chi tiết sau). Qua quan sát, kịch
% bản (\(MIN\), \textbf{M}) là lựa chọn tốt nhất (theo cả ba tiêu chí).
% Thêm vào đó, việc giải quyết một BP cho các đường con trong một lần lặp
% mang lại lợi ích đáng kể. Khá bất ngờ khi kịch bản (\(RAND\),
% \textbf{S}) lại vượt trội cả (\(MED\), \textbf{S}) và (\(MED\),
% \textbf{M}) về thời gian chạy trung bình.

\section{Kết quả thử nghiệm}\label{nhux1eefng-lux1ee3i-uxedch-cux1ee7a-ddd}

\subsection{Phương pháp liệt kê Enum}\label{phuong-phap-enum}
Phương pháp DDD sẽ được so sánh với phương pháp Enum liệt kê BP (từ \cite{foschini2011complexity}).

Trong bài báo gốc, phương pháp Enum được tóm tắt như sau: 

Ý tưởng chính của cách này là chạy thuật toán Dijkstra cho một thời gian khởi hành cụ thể \(t\)
và xây dựng một tập các \emph{chứng nhận}. Mỗi \emph{chứng nhận} này đảm bảo tính chính xác của cây đường đi ngắn nhất được tìm thấy bởi Dijkstra.
Sau đó dần thay đổi thời gian \(t\) và chạy Dijkstra lại để tìm ra các \emph{chứng nhận} mới. Bằng cách liên tục điều chỉnh này, phương pháp khám phá
một cách có hệ thống cho tất cả các cây đường đi ngắn nhất riêng biệt có thể tồn tại với mối thời gian khởi hành khác nhau. 
Dẫn tới việc khám phá toàn diện các khả năng dựa trên ràng buộc thời gian.
Có hai loại \emph{chứng nhận} được sử dụng: \emph{chứng nhận cơ bản} và \emph{chứng nhận tối thiểu}. \emph{Chứng nhận cơ bản} đảm bảo hàm tuyến tính tương ứng với một đường đi duy nhất không thay đổi cho đến thời điểm thất bại của \emph{chứng nhận}; 
sự thất bại của một \emph{chứng nhận cơ bản} tương ứng với một BP tại một cạnh nào đó.
\emph{Chứng nhận tối thiểu} đảm bảo một đường đi nhất định đến một nút nhất định đầu tiên; sự thất bại của một \emph{chứng nhận tối thiểu} tương ứng với một BP tối thiểu trong đồ thị \(D\).
Mỗi \emph{chứng nhận} có một thời điểm thất bại \(t_{fail}\), là thời gian khởi hành sớm nhất sau thời gian hiện tại \(t\) mà \emph{chứng nhận} đó không còn hiệu lực.


\subsection{Kết quả so sánh}\label{ket-qua-so-sanh}

Trong phần này khóa luận sẽ so sánh hiệu năng của thuật toán DDD với thuật
toán Enum liệt kê BP, với kịch bản chọn là \(MED\).



\autoref{table:mdp-result} trình bày kết quả chạy của bài toán MDP với trung bình 10 mẫu được sinh ngẫu
nhiên, có kích thước mạng khác nhau (\(n=30,50\) và khung thời gian
\(T=40\)) cùng các loại hàm vận tốc khác nhau.

% \begin{figure}
% \centering
% \includegraphics{images/Table1.png}
% \caption{Kết quả của MDP}
% \label{table:mdp1}
% \end{figure}

\begin{table}[H]
  \caption{Kết quả của MDP với \(n=[30, 50]\) và \(T=40\)}
  \label{table:mdp-result}
  \resizebox{\textwidth}{!}{
    \begin{tabular}{|p{0.5cm}|p{1cm}|p{1cm}||p{1.5cm}|p{1.5cm}|p{1.3cm}||p{1.5cm}|p{2.5cm}|p{2.5cm}|p{1.3cm}|p{1.3cm}|}
    % \begin{tabular}{SSS|SSSSSSS}
    % \hline
    % \Xhline{2pt}
    \toprule
    n  & g-type & t-type & Avg. BP         & Total BP & \% BP            & Avg. Arcs          & Avg. Time DDD (ms)    & Avg. Time Enum (ms) &  \% Time         & Iters \\ \midrule
    \multirow{2}{*}{30} & \multirow{2}{*}{1} & 1     & 35.2            & 1133     & 3.1             & 5.1               & 49.7                  & 774.7                  & 6.4          & 445        \\
                    &                    & 2     & 44.8            & 1133     & 4.0             & 5.1               & 66.5                  & 856.2                  & 7.8          & 528        \\ \midrule
\multirow{2}{*}{30} & \multirow{2}{*}{2} & 1     & 46.5            & 1133     & 4.1             & 9                 & 32.8                  & 474.9                  & 6.9          & 584        \\
                    &                    & 2     & 46.5            & 1133     & 4.1             & 9                 & 34.2                  & 515.4                  & 6.6          & 596        \\ \midrule
\multirow{2}{*}{30} & \multirow{2}{*}{3} & 1     & 34.2            & 1133     & 3.0             & 6.7               & 33.7                  & 621.1                  & 5.4          & 415        \\
                    &                    & 2     & 45.6            & 1133     & 4.0             & 6.7               & 45.7                  & 667.4                  & 6.8          & 548        \\ \midrule
\multirow{2}{*}{50} & \multirow{2}{*}{1} & 1     & 45.5            & 1913     & 2.4             & 7.6               & 115.6                 & 2375.9                 & 4.9          & 567        \\
                    &                    & 2     & 52.7            & 1913     & 2.8             & 7.5               & 126.8                 & 2496.5                 & 5.1          & 657        \\ \midrule
\multirow{2}{*}{50} & \multirow{2}{*}{2} & 1     & 53.7            & 1913     & 2.8             & 13                & 63.7                  & 1352.1                 & 4.7          & 671        \\
                    &                    & 2     & 56.2            & 1913     & 2.9             & 13                & 68.7                  & 1461.1                 & 4.7          & 690        \\ \midrule
\multirow{2}{*}{50} & \multirow{2}{*}{3} & 1     & 58.9            & 1913     & 3.1             & 9                 & 116.9                 & 1934.2                 & 6.0          & 752        \\
                    &                    & 2     & 49.2            & 1913     & 2.6             & 8.9               & 98.4                  & 2043.5                 & 4.8          & 601         \\ \bottomrule
  \end{tabular}
  }
\end{table}

Các cột trong bảng có ý nghĩa như sau: 
\begin{itemize}
  \item \textbf{n}: số lượng đỉnh trong mạng lưới.
  \item \textbf{g-type}: loại mạng lưới, có ba loại 1,2,3.
  \item \textbf{t-type}: loại hàm vận tốc, có hai loại 1 và 2.
  \item \textbf{Avg. BP}: số BP khám phá được trung bình bằng phương pháp DDD.
  \item \textbf{Total BP}: tổng số BP trong mẫu (phương pháp Enum liệt kê).
  \item \textbf{\% BP}: tỉ lệ phần trăm số BP được khám phá.
  \item \textbf{Avg. Arcs}: số lượng cung trung bình trong nghiệm tối ưu.
  \item \textbf{Avg. Time DDD}: thời gian chạy trung bình của phương pháp DDD (đơn vị ms).
  \item \textbf{Avg. Time Enum}: thời gian chạy trung bình của phương pháp Enum (đơn vị ms).
  \item \textbf{\% Time}: tỉ lệ phần trăm thời gian chạy của DDD với phương pháp Enum.
  \item \textbf{Iters}: số lần lặp trung bình của phương pháp DDD.
\end{itemize}

Lưu ý rằng tổng số BP có thể có trong các mẫu này là
\(|N − 1| \times (T − 1) + 2\), khác với công thức
\(|A| \times (T − 1) + 2\) của \cite{foschini2011complexity}. Vì đối
với các cung có cùng hai đỉnh đầu cuối, ta chỉ cần xử lý một lần.



Kết quả cho thấy phương pháp DDD chỉ cần khám phá một phần nhỏ (từ
\(2.4\%\) đến \(4.1\%\)) tổng số BP, được minh họa bằng \autoref{fig:ddd-enum-bp}. Điều này cho thấy phương pháp này
hoạt động hiệu quả, chỉ tập trung vào các BP tiềm năng thay vì toàn bộ.
Ta cũng có thể thấy rằng khi kích cỡ mẫu tăng lên thì
lợi ích của phương pháp DDD càng rõ rệt, như ở các mẫu \(n=50\) thì phần trăm BP được khám phá
giảm nhiều hơn so với ở các mẫu \(n=30\). Các kết quả chạy với tập mẫu lớn hơn được
trình bày ở \autoref{tab:mdp-res2}, với các giá trị \textbf{\% BP} và \textbf{\% Time} trung bình giảm dần khi mẫu
càng lớn. Ngoài ra thời gian chạy của phương pháp DDD đã cho thấy sự vượt trội so với phương pháp Enum, mô tả ở hình \autoref{fig:ddd-enum}.




\subfile{5.CaseStudy.tex}%

\begin{figure}[]
  \captionbox{So sánh số BP của \(n=[30, 50]\) và \(T=40\)\label{fig:ddd-enum-bp}}{%
    \includegraphics{plots/compare_ddd_enum_bp.png}
  }
  \captionbox{So sánh thời gian chạy của \(n=[30, 50]\) và \(T=40\)\label{fig:ddd-enum}}{%
    \includegraphics{plots/compare_ddd_enum.png}
  }
  % \includegraphics{plots/compare_ddd_enum_bp.png}
  % \caption{So sánh số BP của \(n=[30, 50]\) và \(T=40\)}
  % \label{fig:ddd-enum-bp}
\end{figure}

% \begin{figure}[]
%   \includegraphics{plots/compare_ddd_enum.png}
%   \caption{So sánh thời gian chạy của \(n=[30, 50]\) và \(T=40\)}
%   \label{fig:ddd-enum}
% \end{figure}
\backmatter
\end{document}
% END DOCUMENT
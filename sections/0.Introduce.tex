\documentclass[../main.tex]{subfiles}

% DOCUMENT

\begin{document}

\chapter{Giới thiệu}\label{introduce}


Bài toán tìm đường đi có thời gian đến nhỏ nhất (MATP). Phương pháp đầu
tiên được phát triển để giải quyết vấn đề này dựa trên thuật toán
Bellman-Ford (\cite{bellman1958routing}, \cite{ford2010flows}) và được mô tả bởi
\cite{cooke1966shortest}. Tổng quan về các phương pháp khác cho biến thể
này được đề cập bởi \cite{dean2004shortest}.

MDP được nghiên cứu trong bối cảnh mạng lưới giao thông (\cite{demiryurek2011online}), thậm chí trong phân tích mạng xã hội (\cite{gunturi2012information}).
MTTP đặc biệt quan trọng trong việc tính toán giảm khí thải trong môi
trường đô thị, vì khí thải của một xe hơi trên tuyến đường có thể xấp xỉ
theo tỉ lệ với thời gian di chuyển trên tuyến đường đó; phạm vi tốc độ
giới hạn trong đô thị (công thức khi thải được đề cập ở \cite{jabali2012analysis} với bài toán định tuyến phương tiện phụ thuộc thời gian).

Trong \cite{boland2017continuous} có giới thiệu thuật toán DDD để giải quyết bài
toán tối thiểu hóa chi phí cho mạng dịch vụ theo thời gian liên tục,
thông qua việc sử dụng các biểu thức quy hoạch nguyên trên mạng thời
gian. Giải pháp ở đây là tìm ra các \textbf{thời điểm thích hợp nhất},
cho chúng đi qua các hàm tính toán theo tuần tự nhất định để tạo ra mạng
thời gian từng phần. Các mạng này được thiết kế để luôn giải được, cho
ra một giới hạn kép (cận trên và dưới) cho giá trị tối ưu của thời gian
liên tục. Sau khi tìm được tập thích hợp (số ít các thời điểm), mô hình
quy hoạch nguyên sẽ cho ra kết quả tối ưu.

% Trong bài luận này, tôi trình bày và phân tích thuật toán tìm kiếm
% rời rạc động cho bài toán MDP và bài toán MTTP khi thời gian di chuyển
% trên các cung (cạnh) được xác định bởi các hàm tuyến tính từng đoạn
% (piecewise linear function). Thuật toán sẽ tìm một \textbf{chuỗi} các
% bài toán con (bài toán tìm đường đi ngắn nhất) và giải chúng trong mạng
% thời gian từng phần, cho ra kết quả là cận dưới của bài toán với thời
% gian liên tục. Mỗi đỉnh trong mạng là một cặp (thời gian - vị trí) được
% lấy từ các điểm dừng (breakpoint) của hàm tuyến tính từng đoạn cho việc
% di chuyển.

% \textbf{Chuỗi} kết thúc khi độ dài của đường đi ngắn nhất trong mạng
% thời gian từng phần hiện tại khớp với cận trên tốt nhất.

Các định nghĩa sử dụng trong các phần sau: - \textbf{Cây đường đi ngắn
nhất (Shortest-path tree):} Cho đồ thị \(G=(V,E)\), một cây gốc \(s\)
(\(s\in V\)) trong đó nút \(v\) (\(v\in V\)) thuộc cây thì đường đi từ
nút gốc \(s\) đến nút \(v\) chính là đường đi ngắn nhất.
\backmatter
\end{document}
% END DOCUMENT
\PassOptionsToPackage{unicode}{hyperref}
\PassOptionsToPackage{hyphens}{url}
\PassOptionsToPackage{dvipsnames,svgnames,x11names,table}{xcolor}
\documentclass[fontsize=14pt,DIV=15pt,twoside=false]{scrbook}
% Fix old font command bugs
\DeclareOldFontCommand{\rm}{\normalfont\rmfamily}{\mathrm}
\DeclareOldFontCommand{\sf}{\normalfont\sffamily}{\mathsf}
\DeclareOldFontCommand{\tt}{\normalfont\ttfamily}{\mathtt}
\DeclareOldFontCommand{\bf}{\normalfont\bfseries}{\mathbf}
\DeclareOldFontCommand{\it}{\normalfont\itshape}{\mathit}
\DeclareOldFontCommand{\sl}{\normalfont\slshape}{\@nomath\sl}
\DeclareOldFontCommand{\sc}{\normalfont\scshape}{\@nomath\sc}
\DeclareOldFontCommand{\sfb}{\normalfont\sffamily\bfseries}{\@nomath\sfb}

\usepackage{geometry}
\geometry{a4paper,
    top=2.5cm,
    bottom=2.5cm,
    left=3cm,
    right=2.5cm,
}

\usepackage{amsthm} % For theorem, definition, etc.
\usepackage[at]{easylist}
\usepackage{amsthm}
\newtheorem{theorem}{Định lý} 
\newtheorem{corollary}{Hệ quả}
\newtheorem{lemma}{Bổ đề}
\newtheorem{definition}{Định nghĩa}
\newtheorem{proposition}{Mệnh đề}

% For Vietnamese name of auto-references
\def\algorithmautorefname{Thuật toán}
\def\sectionautorefname{Chương}
\def\chapterautorefname{Chương}
\def\definitionautorefname{Định nghĩa}
\def\propositionautorefname{Mệnh đề}
\def\lemmaautorefname{Bổ đề}
\def\corollaryautorefname{Hệ quả}
\def\theoremautorefname{Định lý}

% To mark the word and create a label for it
\makeatletter
\newcommand{\setword}[2]{%
  \phantomsection
  #1\def\@currentlabel{\unexpanded{#1}}\label{#2}%
}
\makeatother

\AtBeginDocument{%
\renewcommand{\figureautorefname}{Hình}
\renewcommand\subfigureautorefname{(\alph{subfigure})}
\renewcommand{\tableautorefname}{Bảng}
}

\usepackage[%
    font={small,sf},
    labelfont=bf,
    format=hang,    
    format=plain,
    margin=0pt,
    width=0.8\textwidth,
]{caption}
\usepackage{subcaption} % For subfigure

% vnthesis mới đã có phần package pseudocode.
% Nằm trong init.
% PSEUDO-CODE
\usepackage{algorithm}
\makeatletter
\renewcommand{\ALG@name}{Thuật toán} % Algorithm -> Thuật toán
\makeatother
\usepackage[noEnd=false,indLines=true]{algpseudocodex}

\newcommand{\And}{\textbf{and~}}
\newcommand{\Or}{\textbf{or~}}
\newcommand{\Xor}{\textbf{xor~}}
\newcommand{\Not}{\textbf{not~}}
\newcommand{\To}{\textbf{to~}}
\newcommand{\DownTo}{\textbf{downto~}}
\newcommand{\True}{\textbf{true~}}
\newcommand{\False}{\textbf{false~}}
\newcommand{\Input}{\item[\textbf{Input:}]}
\renewcommand{\Output}{\item[\textbf{Output:}]}
\newcommand{\Print}{\State \textbf{print~}}
\renewcommand{\Return}{\State \textbf{return~}}
% END PSEUDO-CODE.

\usepackage{tabularx} % For dynamic table
\usepackage{multirow} % For multirow in table
\usepackage{parcolumns} % For parallel text
\usepackage{lscape} % For landscape page

\usepackage{lscape} % For landscape page

% For include other files
\usepackage{subfiles} % Best loaded last in the preamble

\usepackage{makecell} % Change the thickness of the line in table

\usepackage{iftex}

\usepackage[utf8]{vietnam}
\usepackage[vietnamese=nohyphenation]{hyphsubst}
\usepackage[utf8]{inputenc}
\usepackage[vietnamese]{babel}

% Biblo
\usepackage[
    style=chem-angew,
    sorting=nyt,
    maxbibnames=99
    ]{biblatex}
\addbibresource{refs.bib}
\usepackage{csquotes} % biblatex recommended

\usepackage{parcolumns} % For parallel text

% For smaller column in table
\newcolumntype{s}{>{\hsize=.5\hsize}X}

\usepackage{fontspec}
\usepackage{pifont}
\setmainfont{Times New Roman}[
    ]
\setmonofont{JetBrainsMono}[
        Path = jetbrains-mono/,
        Extension = .ttf,
        UprightFont = *-Regular,
        ItalicFont = *-Italic,
        BoldFont = *-Bold,
        BoldItalicFont = *-BoldItalic,
    ]
    
\usepackage{fontawesome5}
\usepackage{unicode-math}
\usepackage{amsmath}
\usepackage{utfsym}
\usepackage{pmboxdraw}

\usepackage{xcolor}
\definecolor{deepblue}{RGB}{0, 0, 112}
\definecolor{cyclamen}{RGB}{255, 145, 237}
\definecolor{bubblegum}{RGB}{255, 115, 232}
\definecolor{silver}{RGB}{208,208,208}

\usepackage{listings}
\lstset{
    aboveskip=5mm,
    belowskip=5mm,
    xleftmargin=1.5em,
    showstringspaces=false,
    columns=flexible,
    numbers=left,
    numberfirstline,
    stepnumber=5,
    numberstyle=\small\color{gray},
    keywordstyle=\bfseries\color{deepblue},
    commentstyle=\itshape\color{deepgreen},
    stringstyle=\color{bubblegum},
    emphstyle=\color{deepred},
    keepspaces=true,
    breaklines=true,
    breakatwhitespace=false,
    breakautoindent=true,
    frame=single,
    framesep=3mm,
    rulecolor=\color{silver},
    tabsize=2,
    texcl=true,
    captionpos=b
}

\usepackage[]{hyperref}
\usepackage{bookmark}
\hypersetup{
    colorlinks=true,
    linkcolor=black,
    filecolor=bubblegum,
    urlcolor=deepblue,
    pdftitle={Phương pháp rời rạc động giải các bài toán \\ tìm đường đi
có yếu tố thời gian},
}

% from pandoc
\newcommand{\passthrough}[1]{#1}
\usepackage{graphicx}
\makeatletter
\def\maxwidth{\ifdim\Gin@nat@width>\linewidth\linewidth\else\Gin@nat@width\fi}
\def\maxheight{\ifdim\Gin@nat@height>\textheight\textheight\else\Gin@nat@height\fi}
\makeatother
% Scale images if necessary, so that they will not overflow the page
% margins by default, and it is still possible to overwrite the defaults
% using explicit options in \includegraphics[width, height, ...]{}
\setkeys{Gin}{width=\maxwidth,height=\maxheight,keepaspectratio}
% Set default figure placement to htbp
\makeatletter
\def\fps@figure{htbp}
\makeatother
\setlength{\emergencystretch}{3em} % prevent overfull lines
\providecommand{\tightlist}{\setlength{\itemsep}{\smallskipamount}\setlength{\parskip}{\smallskipamount}}
% definitions for citeproc citations

% end from pandoc

\usepackage{fancyhdr}
\pagestyle{fancy}
\fancyhf{}
% \fancyhead[L]{\includegraphics[height=1cm,keepaspectratio]{logo/page-header-logo.png}} 
\fancyhead[R]{\small\leftmark} \fancyfoot[C]{\thepage}
\fancypagestyle{plain}{%
\fancyhf{}
% \fancyhead[L]{\includegraphics[height=1cm,keepaspectratio]{logo/page-header-logo.png}} 
\fancyhead[R]{\small\leftmark} \fancyfoot[C]{\thepage}
}

\usepackage{titlesec}
\setcounter{tocdepth}{2}
\renewcommand{\baselinestretch}{1.3}
\usepackage{indentfirst}
\setlength{\parindent}{14pt}
\setlength{\parskip}{6pt}

\setcounter{secnumdepth}{4}
\renewcommand{\thechapter}{\arabic{chapter}}
\renewcommand{\thesection}{\thechapter.\arabic{section}}
\renewcommand{\thesubsection}{\thesection.\arabic{subsection}}
\renewcommand{\thesubsubsection}{\thesubsection.\arabic{subsubsection}}
\renewcommand{\paragraph}{\arabic{paragraph}}
\renewcommand{\subparagraph}{\alph{subparagraph}}
\titleformat{\chapter}
{\bfseries\fontsize{20}{20}\selectfont}
{\thechapter.}{1em}{}

\titleformat{\section}
{\bfseries\fontsize{18}{20}\selectfont}
{\thesection}{1em}{}

\titleformat{\subsection}
{\bfseries\fontsize{16}{18}\selectfont}
{\thesubsection}{1em}{}

\titleformat{\subsubsection}
{\passthrough{\lstinline!\\bfseries\\fontsize\{16\}\{16\}\\selectfont!}}
{\thesubsubsection}{1em}{}

\titleformat{\paragraph}
{\hspace{1em}\fontsize{14}{16}\selectfont\rmfamily}
{\paragraph}{1em}{}

\titleformat{\subparagraph}
{\hspace{2em}\fontsize{14}{16}\selectfont\rmfamily}
{\subparagraph}{1em}{}

\usepackage{footnotebackref}

%begin tables-vrules.lua
\usepackage{longtable,booktabs,array}
\usepackage{calc} % for calculating minipage widths
% Correct order of tables after \paragraph or \subparagraph
\usepackage{etoolbox}
\makeatletter
\patchcmd\longtable{\par}{\if@noskipsec\mbox{}\fi\par}{}{}
\makeatother
% Allow footnotes in longtable head/foot
\IfFileExists{footnotehyper.sty}{\usepackage{footnotehyper}}{\usepackage{footnote}}
\makesavenoteenv{longtable}
\setlength{\aboverulesep}{0pt}
\setlength{\belowrulesep}{0pt}
\renewcommand{\arraystretch}{1.3}
%end tables-vrules.lua
\usepackage{multicol}
  \newlength{\currentparskip}

\usepackage{environ}
\NewEnviron{centerboxed}{
\makebox[\linewidth][c]{%
\begin{minipage}[t][0.9\textheight][t]{\linewidth}%
\BODY
\end{minipage}%
}%
}

\NewEnviron{sign}{
\hfill\begin{tabular}{c}
{\it Hà Nội, ngày 19 tháng 5 năm 2024}
\\ \BODY
\\ \\[1cm]
{\bf Bùi Khánh Duy}\\
\end{tabular}%
}

\newcommand{\toc}[0]{
    % change line spacing
\renewcommand{\baselinestretch}{1.0}
\tableofcontents
\listoffigures
\listoftables
}

\usepackage{tikz}
\usetikzlibrary{calc}
\newcommand{\frontmatterthesisframe}[2]{
\begin{tikzpicture}[remember picture,overlay]
\centering
\ifnum#1=0
#2
\fi

\ifnum#1=1
\draw[blue!70!black,line width=4pt]
([xshift=-1.5cm,yshift=-2cm]current page.north east) coordinate (A)--
([xshift=3.5cm,yshift=-2cm]current page.north west) coordinate(B)--
([xshift=3.5cm,yshift=2cm]current page.south west) coordinate (C)--
([xshift=-1.5cm,yshift=2cm]current page.south east) coordinate(D)--cycle;
\draw
([xshift=-0.5cm,yshift=0.5cm]A)--
([xshift=0.5cm,yshift=0.5cm]B)--
([xshift=0.5cm,yshift=-0.5cm]B)--
([xshift=-0.5cm,yshift=-0.5cm]B)--
([xshift=-0.5cm,yshift=0.5cm]C)--
([xshift=0.5cm,yshift=0.5cm]C)--
([xshift=0.5cm,yshift=-0.5cm]C)--
([xshift=-0.5cm,yshift=-0.5cm]D)--
([xshift=-0.5cm,yshift=0.5cm]D)--
([xshift=0.5cm,yshift=0.5cm]D)--
([xshift=0.5cm,yshift=-0.5cm]A)--
([xshift=-0.5cm,yshift=-0.5cm]A)--
([xshift=-0.5cm,yshift=0.5cm]A);
\draw
([xshift=0.3cm,yshift=-0.3cm]A)--
([xshift=-0.3cm,yshift=-0.3cm]B)--
([xshift=-0.3cm,yshift=0.3cm]B)--
([xshift=0.3cm,yshift=0.3cm]B)--
([xshift=0.3cm,yshift=-0.3cm]C)--
([xshift=-0.3cm,yshift=-0.3cm]C)--
([xshift=-0.3cm,yshift=0.3cm]C)--
([xshift=0.3cm,yshift=0.3cm]D)--
([xshift=0.3cm,yshift=-0.3cm]D)--
([xshift=-0.3cm,yshift=-0.3cm]D)--
([xshift=-0.3cm,yshift=0.3cm]A)--
([xshift=0.3cm,yshift=0.3cm]A)--
([xshift=0.3cm,yshift=-0.3cm]A);
\fi

\ifnum#1=2
\draw [line width=2pt]
([xshift=3.5cm,yshift=-2.5cm]current page.north west)
rectangle
([xshift=-2cm,yshift=2.5cm]current page.south east);
\draw [line width=0.5pt]
([xshift=3.6cm,yshift=-2.6cm]current page.north west)
rectangle
([xshift=-2.1cm,yshift=2.6cm]current page.south east);
\fi

\ifnum#1=3
\draw[line width = 3pt]
([xshift=3.5cm,yshift=-2.5cm]current page.north west)
rectangle
([xshift=-2cm,yshift=2.5cm]current page.south east);
\fi
\end{tikzpicture}
}

\newcommand{\frontmatterthesisinfo}[2]{
\ifnum#1=0
#2
\fi

\ifnum#1=1
\begin{centerboxed}
\centering
\vspace*{3mm}
\fontsize{14}{16}\selectfont Đại học Quốc gia Hà Nội \\
\fontsize{14}{16}\selectfont Trường Đại học Khoa học Tự nhiên \\
\fontsize{13}{16}\selectfont\textbf{Khoa Toán - Cơ - Tin học} \\
\fontsize{8}{16}\Pisymbol{dingbat}{69} \hspace{1.4cm}  \Huge\usym{1F56E} \hspace{1cm} \fontsize{8}{16}\Pisymbol{dingbat}{70}\\
\vspace*{1.5cm}
% \includegraphics[height=3cm,keepaspectratio]{"logo/logo.jpg"}\\
\vspace*{1.5cm}
\fontsize{18}{1}\selectfont \textbf{Phương pháp rời rạc động giải các
bài toán \\ tìm đường đi có yếu tố thời gian} \\
\vspace{2cm}
{\fontsize{14}{1}\selectfont Khóa luận tốt nghiệp}  \\
{\fontsize{14}{1}\selectfont Khóa luận tốt nghiệp đại học hệ chính
quy}  \\
{\fontsize{14}{1}\selectfont Ngành: Khoa học máy tính và thông tin}  \\
\vspace{2cm}
{\fontsize{14}{1}\selectfont\textbf{
\begin{tabular}{ll}
Người hướng dẫn:  & TS. Vũ Đức Minh
\end{tabular}
}}
\\
\vspace{2cm}
\raggedright{\hspace{4.2cm}\fontsize{14}{1}\selectfont \textbf{Người thực hiện:}\\
\vspace{0.5cm}\hspace{5.5cm}
\begin{tabular}{ll}
Bùi Khánh Duy &  \\ \\[0.25em] 
\end{tabular}
}
\vfill
\centering
\fontsize{14}{14}\selectfont{\textbf{Hà Nội - 2024}}
\end{centerboxed}
\fi

\ifnum#1=2
\begin{centerboxed}
\centering
\vspace*{3mm}
{\fontsize{13}{1}\selectfont Đại học Quốc gia Hà Nội \\
Trường Đại học Khoa học Tự nhiên \\
\textbf{Khoa Toán - Cơ - Tin học}
}\\
\vspace{3.5cm}
{\fontsize{14}{1}\selectfont \textbf{
\begin{tabular}{ll}
Bùi Khánh Duy \\ \\[0.25em] 
\end{tabular}
}}
\\
\vspace{2cm}

{\fontsize{18}{1}\selectfont \textbf{Phương pháp rời rạc động giải các
bài toán \\ tìm đường đi có yếu tố thời gian} }
\\\vspace{5cm}

{\fontsize{14}{1}\selectfont Khóa luận tốt nghiệp đại học hệ chính quy}  \\
{\fontsize{14}{1}\selectfont Ngành: Khoa học máy tính và thông tin}  \\
{\fontsize{14}{1}\selectfont (Chương trình đào tạo chuẩn)}  \\

\vfill
\centering
\fontsize{14}{14}\selectfont{\textbf{Hà Nội - 2024}}
\end{centerboxed}
\fi

\ifnum#1=3
\begin{centerboxed}
\centering
\vspace*{3mm}
{\fontsize{13}{1}\selectfont ĐẠI HỌC QUỐC GIA HÀ NỘI \\
TRƯỜNG ĐẠI HỌC KHOA HỌC TỰ NHIÊN \\
\textbf{Khoa Toán - Cơ - Tin học}
}\\
\vspace{3.5cm}
{\fontsize{14}{1}\selectfont \textbf{
\begin{tabular}{ll}
Bùi Khánh Duy \\ \\[0.25em] 
\end{tabular}
}}
\\
\vspace{2cm}

{\fontsize{18}{1}\selectfont \textbf{Phương pháp rời rạc động giải các
bài toán \\ tìm đường đi có yếu tố thời gian} }
\\\vspace{5cm}

{\fontsize{14}{1}\selectfont Khóa luận tốt nghiệp đại học hệ chính quy}  \\
{\fontsize{14}{1}\selectfont Ngành: Khoa học máy tính và thông tin}  \\
{\fontsize{14}{1}\selectfont (Chương trình đào tạo chuẩn)}  \\

\vspace{2cm}
{\fontsize{14}{1}\selectfont \textbf{
\begin{tabular}{ll}
Cán bộ hướng dẫn:  & TS. Vũ Đức Minh
\end{tabular}
}}
\vfill
\centering
\fontsize{14}{14}\selectfont{\textbf{Hà Nội - 2024}}
\end{centerboxed}
\fi

}
% fix toc use san serif font in header
\setkomafont{chapter}{\normalfont\bfseries\large}
\RedeclareSectionCommands[
tocentryformat=\usekomafont{chapter},
tocpagenumberformat=\usekomafont{chapter}
]{chapter}

% DOCUMENT

\begin{document}
\frontmatter
\pagestyle{empty}
\begin{titlepage}
\frontmatterthesisframe{2}{}
\frontmatterthesisinfo{2}{}
\end{titlepage}

\begin{titlepage}
\frontmatterthesisframe{2}{}
\frontmatterthesisinfo{3}{}
\end{titlepage}

\mainmatter
\pagestyle{fancy}

\chapter*{Lời cảm ơn}\label{cam-on}
\addcontentsline{toc}{chapter}{Lời cảm ơn}

\markboth{Lời cảm ơn}{Lời cảm ơn}
Trong những năm tháng sinh viên, em đã được học hỏi và trải qua nhiều điều, trong đó có những khó khăn và thách thức. 
Em xin bày tỏ lòng biết ơn sâu sắc đến bạn bè, thầy cô và gia đình đã luôn ở bên cạnh, hỗ trợ và giúp đỡ em trong suốt thời gian qua. 
Các thầy cô ở khoa Toán - Cơ - Tin học, trường Đại học Khoa học Tự nhiên, ĐHQG Hà Nội đã luôn tạo điều kiện tốt nhất trong quá trình học tập và phát triển. 

Khóa luận "\textbf{Phương pháp rời rạc động giải các bài toán tìm đường đi có yếu tố thời gian}" của em đã không thể hoàn thành mà không có sự hướng dẫn, 
giúp đỡ tận tình của thầy \textbf{TS. Vũ Đức Minh}. 
Em xin chân thành cảm ơn thầy đã dành thời gian để chỉ bảo em tận tình trong quá trình thực hiện khóa luận này. Cuối cùng, em xin chân thành cảm ơn tất cả mọi người đã giúp đỡ em trong suốt thời gian qua.

Mặc dù đã có nhiều cố gắng hoàn thiện nhưng do hạn chế ở kĩ năng và kinh nghiệm của bản thân nên sẽ không tránh khỏi những thiếu sót còn tồn đọng. Em rất mong nhận được sự góp ý, chỉ bảo từ quý thầy cô và các bạn để khóa luận của em trở nên hoàn thiện hơn.

\emph{Em xin chân thành cảm ơn!}


\begin{sign}

    % Bùi Khánh Duy
    
    \end{sign}

\chapter*{Mở đầu}\label{lux1eddi-mux1edf-ux111ux1ea7u}
\addcontentsline{toc}{chapter}{Mở đầu}

\markboth{Mở đầu}{Mở đầu}

\textbf{Tìm đường đi ngắn nhất trong mạng} vốn là một trong những bài toán tối ưu cơ bản. Tuy nhiên những biến thể  của nó vẫn luôn hấp dẫn và có ý nghĩa thực tiễn cao. Các ứng dụng của bài toán này rất đa dạng, từ việc tìm đường đi cho mô hình mạng lưới giao thông đến các bài toán vận tải.
Ngoài ra, đường đi ngắn nhất không dừng lại ở khoảng cách, nó còn có thể là nhỏ nhất về thời gian, chi phí, hoặc một tiêu chí nào đó khác. 

Khóa luận sẽ tập trung vào bài toán thời gian di chuyển nhỏ nhất, với việc thời gian di chuyển trên một cung (trọng số của cung) trong mạng phụ thuộc vào \textbf{thời gian khởi hành} di chuyển trên cung đó. Điều này có ý nghĩa thực tiễn, khi không phải lúc nào việc đến đích càng sớm càng tốt cũng mục tiêu duy nhất. 
Các mục tiêu khác nhau như giảm thiểu sự chênh lệch giữa thời gian đến đích và thời gian khởi hành, hay giảm thiểu tổng thời gian di chuyển theo đường đi từ nguồn đến đích đều là những đáng để xem xét.
Phương pháp được giới thiệu là thuật toán khám phá rời rạc động (DDD), với thời gian di chuyển là các hàm vận tốc phụ thuộc vào thời gian, tuyến tính từng khúc. 
Thuật toán hoạt động trên các mạng thời gian (TEN) trong đó chi phí của cung sẽ đại diện bởi các cận dưới về thời gian di chuyển trong khoảng thời gian tiếp theo. 
Một đường đi ngắn nhất trong mạng TEN này tạo ra cận dưới và cận trên cho giá trị của nghiệm tối ưu. 
Thuật toán liên tục tinh chỉnh sự rời rạc bằng cách khai thác các thời điểm dừng (BP) của các hàm vận tốc, loại bỏ các khoảng thời gian thừa, cải thiện cận dưới và cận trên cho đến khi, trong một số lần lặp hữu hạn, cả hai cận hội tụ (thu được nghiệm tối ưu). 
Các thử nghiệm cho thấy chỉ có một phần nhỏ các BP cần được khám phá, và tỷ lệ này giảm đi khi độ dài của khung thời gian và kích thước của mạng tăng lên, làm cho các thuật toán vô cùng hiệu quả và có khả năng mở rộng cao.

Bố cục của khóa luận như sau:

\begin{itemize}
    \item \autoref{introduce}: Giới thiệu.\\
        Trình bày một số khái niệm, phát biểu bài toán và các công trình liên quan.
    \item \autoref{giux1ea3i-buxe0i-touxe1n-mdp}: Bài toán MDP. \\
        Trình bày bài toán tìm đường đi có thời gian di chuyển nhỏ nhất phụ thuộc vào hàm vận tốc (MDP), mô hình hóa và giải bài toán thông qua mã giả và ví dụ minh họa.
    
    % \item \autoref{giux1ea3i-buxe0i-touxe1n-mttp}: Bài toán tìm đường đi ngắn nhất theo thời gian thực hiện.\\
    %     Trình bày bài toán tìm đường đi ngắn nhất theo thời gian thực hiện, mô hình hóa và giải bài toán tìm đường đi ngắn nhất theo thời gian thực hiện.
    
    \item \autoref{cuxe1c-thux1eed-nghiux1ec7m}: Các thử nghiệm.\\
        Trình bày các thử nghiệm và kết quả chạy của thuật toán DDD và ví dụ thực tế.
    
    \item \autoref{kux1ebft-luux1eadn}: Kết luận.\\
        Tóm tắt kết quả và đề xuất hướng phát triển tiếp theo.
    \item \autoref{appendix}: Phụ lục.\\
        Bao gồm các thông tin bổ sung, các hàm vận tốc cho ví dụ ở \autoref{giux1ea3i-buxe0i-touxe1n-mdp} và toàn bộ kết quả chạy.

\end{itemize}

\textbf{Từ khoá:} Dynamic Discretization Discovery, Time-dependent Shortest Path Problem, Minimum Duration Time-Dependent Path Problem.

% Thêm phần bảng viết tắt.
\chapter*{Bảng ký hiệu}\label{bang-ky-hieu}
\addcontentsline{toc}{chapter}{Bảng ký hiệu}

\markboth{Bảng ký hiệu}{Bảng ký hiệu}

\begin{table}[h]
    \centering
    \small
    % set array stretch to 1.0
    \renewcommand{\arraystretch}{1.1}
    \begin{tabularx}{\textwidth}{|p{1.5cm}|X|X|}
    \toprule
    \textbf{Kí hiệu} & \textbf{Tên đầy đủ}                               & \textbf{Ý nghĩa}                                                                    \\ \midrule
    % \((i,t)-MG\)         & \((i,t)-mangrove\)                                & Là \(i-mangrove\) cho điểm dừng \((i, t)\)                                          \\ \midrule
    % AMG              & Arc-completed Mangrove                            & Mangrove hoàn chỉnh                                                                 \\ \midrule
    % \(i-MG\)             & \(i-mangrove\)                                    & Hợp của \(i-FSPT\) và \(i-BSPT\) cho cùng một điểm dừng tại đỉnh                    \\ \midrule
    ABSPT            & Arc-completed Backwards Shortest Path Tree        & Cây đường đi ngắn nhất hoàn chỉnh lùi                                               \\ \midrule
    AFSPT            & Arc-completed Forwards Shortest Path Tree         & Cây đường đi ngắn nhất hoàn chỉnh tiến                                              \\ \midrule
    BP               & Breakpoint                                        & Thời điểm dừng: thời điểm có thời gian nguyên.                                     \\ \midrule
    BSPT             & Backwards Shortest Path Tree                      & Cây đường đi ngắn nhất lùi                                                          \\ \midrule
    DDD              & Dynamic Discretization Discovery                  & Khám phá rời rạc động                                                               \\ \midrule
    FIFO             & First in first out                                & Vào trước ra trước                                                                  \\ \midrule
    FSPT             & Forwards Shortest Path Tree                       & Cây đường đi ngắn nhất tiến                                                         \\ \midrule
    MATP             & Minimum Arrival Time Path Problem                 & Bài toán Đường đi thời gian đến tối thiểu                                           \\ \midrule
    MD               & Minimum Duration                                  & Thời gian di chuyển tối thiểu (từ lúc khởi hành đến lúc tới đích)                   \\ \midrule
    MDP              & Minimum Duration Time-Dependent Path Problem      & Bài toán Đường đi thời gian di chuyển tối thiểu                                     \\ \midrule
    % MTTP             & Minimum Travel Time Time-Dependent Path Problem   & Bài toán Đường Đi Ngắn Nhất Theo Thời Gian thực hiện                                \\ \midrule
    SSSP             & Single-source shortest path                       & Tìm đường đi ngắn nhất xuất phát từ một đỉnh, sử dụng Dijkstra với hàng đợi ưu tiên \\ \midrule
    TDSP             & Time-dependent shortest path                      & Đường đi ngắn nhất phụ thuộc thời gian                                              \\ \midrule
    TDSPP            & Time-Dependent Shortest Path Problem              & Bài toán tìm đường đi ngắn nhất phụ thuộc thời gian                                 \\ \midrule
    TEN              & Time-expanded network                             & Mạng thời gian                                                                      \\ \midrule
    % TENL             & Time-expanded network with associated arc lengths & Mạng thời gian có độ dài của cung                                                   \\ \midrule
    TT               & Travel time                                       & Thời gian di chuyển                                                                 \\ \midrule
    UTT              & Underestimated travel time                        & Cận dưới của thời gian di chuyển                                                    \\ \bottomrule
    \end{tabularx}
\end{table}

% Change line spacing back to 1 in toc only

\toc

\subfile{sections/1.Introduce}%
\subfile{sections/2.MDP}%
% \subfile{sections/3.MTTP}%
\subfile{sections/4.Computational}%
% \subfile{sections/5.CaseStudy}%

% \chapter*{Kết luận}\label{kux1ebft-luux1eadn}
% \addcontentsline{toc}{chapter}{Kết luận}

\subfile{sections/7.Remark.tex}

\printbibliography
\subfile{sections/6.Appendix}%

\backmatter
\end{document}
% END DOCUMENT